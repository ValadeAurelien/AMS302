\documentclass[11pt,a4paper]{article}

\usepackage[utf8]{inputenc}
\usepackage[francais]{babel}
\usepackage[T1]{fontenc}
\usepackage{amsmath}
\usepackage{amsfonts}
\usepackage{amssymb}
\usepackage[colorlinks=false]{hyperref} % liens dans le sommaire
% \usepackage[top = 2.54cm, bottom = 2.54cm, left = 2.54cm, right = 2.54cm]{geometry}
\usepackage[top = 2cm, bottom = 2cm, left = 2cm, right = 2cm]{geometry}
\usepackage{graphicx}
\usepackage{caption}
\hypersetup{colorlinks, linkcolor = black, citecolor = black} % enlève couleur des liens
\usepackage{hyperref}
\usepackage{color}
\frenchbsetup{StandardLists=true}
\usepackage{float}
\usepackage{fancyhdr}
\usepackage{mathtools}
\pagestyle{fancy}

\newcommand{\dx}[1]{\dfrac{\partial #1}{\partial x}}
\newcommand{\norm}[1]{\big|\big|#1\big|\big|}
\newcommand{\question}[2]{\paragraph{Question #1 --}\hspace{-7pt}\textit{#2} \\}
\newcommand{\tphi}{\widetilde{\Phi}}
\newcommand{\intsigma}{\widetilde{\Sigma}}
\newcommand{\F}{\mathcal{F}}
\newcommand{\Qt}{\widetilde{Q}}
\newcommand{\Phit}{\widetilde{\Phi}}


\begin{document}

\begin{titlepage}
  ~\vspace{90pt}
  \centering \bfseries

  \huge Rapport de TP \\AMS302

  \vspace{50pt}
  \rule{0.5\textwidth}{1pt}
  \vspace{50pt}

  \Huge Solveur Déterministe \\ pour le transport de particules neutres

  \vspace{50pt}
  \rule{0.5\textwidth}{1pt}

  \vspace{50pt}
  \huge {\itshape Benoît Sohet \\ \& \\ Aurélien Valade}

  \vfill
  \begin{tabular}{cc}
    \begin{minipage}{.49\textwidth}
      \centering
      %\includegraphics[height=0.1\textheight]{logo_ups}
    \end{minipage}
    &
      \begin{minipage}{.49\textwidth}
        \centering
        %\includegraphics[height=0.1\textheight]{logo_ensta}
      \end{minipage}
  \end{tabular}

\end{titlepage}

\newpage

\section*{Introduction}

Dans ce TP on cherche à résoudre le problème différentiel suivant 
\begin{equation}
  \label{eq:principal}
  \begin{cases}
    \mu \dx{\Phi}(x, \mu) + \Sigma_t(x) \Phi(x, \mu) =
    \dfrac{1}{2} \Sigma_s(x) \int_1^{-1} \Phi(x, \mu') d\mu' + S(x, \mu) & \forall (x, \mu) \in [0,1]\times[-1,1]  \medskip\\ 
    \mu n(x) < 0  & \forall (x, \mu) \in \{0,1\}\times[-1,1] 
  \end{cases}
\end{equation}

\begin{itemize}
\item $\Phi(x, \mu)$ le flux neutronique
\item $\Sigma_t$ la section efficace totale 
\item $n(x)$ la normale sortante : $n(0) = -1, ~ n(1) = 1$
\end{itemize}

On fixe pour cela deux types de conditions aux limites 
\begin{itemize}
\item Flux entrant à gauche
  \begin{equation}
    \begin{cases}
      \Phi^{-} (0, \mu) = \frac{1}{\mu}  ~~~ \forall \mu \in [0,1]\\
      \Phi^{-} (1, \mu) = 0 ~~~ \forall \mu \in [-1,0]
    \end{cases}
  \end{equation}
\item Source nulle ou unitaire
  \begin{align}
    S(x) &= 0 ~~~ \forall x \in [0,1] \\
    S(x) &= 1 ~~~ \forall x \in [0,1] 
  \end{align}
\end{itemize}	

On utilisera dans la suite l'erreur pour une fonction $f$ et son approximation $\tilde{f}$ : 
\begin{equation}
  e_{L^2}(f, \tilde{f}) = \frac{\norm{f-\tilde{f}}_{L_2}}{\norm{f}_{L_2}}
\end{equation}

\emph{L'utilisation des codes nécessite gnuplot ainsi qu'un système d'exploitation respectant la norme POSIX.}

\section{Matériau purement absorbant}

\subsection{Matériau homogène - courant entrant unitaire}

\question{9.1}{Rappeler la solution analytique à ce problème avec source nulle et flux entrant à gauche, et un matériau purement absorbant et homogène.}

L'équation à résoudre s'écrit alors :
\begin{equation}
  \begin{cases}
    \mu \dx{\Phi}(x, \mu) + \Sigma_a \Phi(x, \mu) = 0 & \forall (x, \mu) \in [0,1]\times[-1,1]  \medskip\\ 
    \Phi(0,\mu) = \frac{1}{\mu} & \forall \mu \in [0,1] \medskip\\
    \Phi(1,\mu) = 0 & \forall \mu \in [-1,0]
  \end{cases}
  \label{eq:pasdiff}
\end{equation}

La solution est donc :
\begin{equation}
 \Phi(x, \mu) = A(\mu) e^{-\frac{\Sigma_a}{\mu}x} ,
\end{equation}

avec $A(\mu)$ qui vérifie les conditions initiales :
\begin{equation}
 A(\mu) = 
 \begin{cases}
  \frac{1}{\mu} & \text{si } \mu \in [0,1] \\
  0             & \text{si } \mu \in [-1,0] .
 \end{cases}
\end{equation}

On peut remarquer que l'on retrouve exactement la même solution que si l'on avait imposé un flux entrant nul des deux côtés, et une source sous forme de dirac en 0 lorsque $\mu>0$.

\question{9.2}{Implémenter un solveur déterministe permettant d'évaluer le flux neutronique en chaque point de la géométrie. Ce solveur sera basé sur la méthode ``diamant'' pour la discrétisation spatiale.} 

\paragraph{Notations :} On note par $(x_i)$ la segmentation choisie pour $[0,1]$, $I_i = [x_i,x_{i+1}]$ et $\delta_i = x_{i+1} - x_i$.
On peut alors écrire $\phi_i( \mu) = \phi(x_i,\mu)$.
La méthode ``diamant'' fait l'hypothèse suivante :

\begin{equation}
 \forall (\mu,x) \in [-1,1]\times I_i, \quad \phi(x,\mu) \simeq \frac{\phi_i(\mu) + \phi_{i+1}(\mu)}{2} .
\end{equation}
Pour simplifier les expressions suivantes, on se fixe un $\mu \in  [-1,1]$ et on écrit directement $\phi_i(\mu) = \phi_i$.
De là, l'équation~\ref{eq:pasdiff} devient (on rajoute un terme source $S$ pour ne pas avoir à refaire les calculs pour la prochaine question) :

\begin{equation}
\begin{align}
 & \mu \frac{\phi_{i+1}-\phi_i}{\Delta_i} + \Sigma_t \frac{\phi_{i+1}+\phi_i}{2} = S\\
 \Leftrightarrow \quad & 2 \mu (\phi_{i+1}- \phi_i) + \Delta_i \Sigma_t (\phi_{i+1}+\phi_i) = 2 \Delta_i S\\
 \Leftrightarrow \quad & \eta^+ \phi_{i+1} - \eta^- \phi_i = 2 \Delta_i S &&  \text{avec } \eta^{\pm} = 2 \mu \pm \Delta_i \Sigma_t .
 \end{align}
\end{equation}

Or, suivant le signe de $\mu$, on n'isolera pas le même terme dans cette équation.
En effet, si $\mu>0$, les conditions de flux entrant ne donnent une information qu'en $x=0$ : il faut donc utiliser cette relation récurrente afin d'avoir $\phi_{i+1}$ à partir de $\phi_i$, ce qui donne :
\begin{equation}
 \phi_{i+1} = \frac{\eta^- \phi_i + 2 \Delta_i S}{\eta^+}
 \label{eq:recu}
\end{equation}

Et inversement lorsque $\mu<0$.
De plus on remarque que :
\begin{equation}
 \frac{\eta^+}{\eta^-} = \frac{- \eta^+}{-\eta^-} = \frac{-2 \mu - \Delta_i \Sigma_t}{-2 \mu + \Delta_i \Sigma_t} = \frac{2 |\mu| - \Delta_i \Sigma_t}{2 |\mu| + \Delta_i \Sigma_t} ,
\end{equation}

donc si l'on change la définition des $\eta$ en $\eta^{\pm} = 2|\mu| \pm \Delta_i \Sigma_t$, on retrouve pour $\mu<0$ la même formulation~\ref{eq:recu} que pour $\mu>0$ :
\begin{equation}
 \phi_i = \frac{\eta^- \phi_{i+1} + 2 \Delta_i S}{\eta^+} .
\end{equation}
Il faut simplement faire attention à bien retourner le vecteur $(\phi_i)$.

\subsection{Matériau homogène - source uniforme}

\question{10}{Reprendre la démarche de la question 9 pour ces nouvelles conditions. On veillera à ce que les fonctionnalités
 ajoutées dans le solveur pour le traitement du terme source restent compatibles avec le traitement du flux entrant imposé nécessaire pour la question 9.}
 Coucou
 \subsection{Matériau non homogène - courant entrant unitaire}
 
 On se place maintenant dans le cas d'une source nulle et de conditions de flux entrant imposé, pour un matériau non homogène :
\begin{equation}
  \Sigma_a =
  \begin{cases}
    1 &\mbox{si } x<0.3 \\
    3 &\mbox{si } 0.3<x<0.7 \\
    1 &\mbox{si } 0.7<x \\
  \end{cases}
\end{equation}

\question{11}{Reprendre la démarche de la question 9 pour ces nouvelles conditions.
On s'attachera à ce que les fonctionnalités ajoutées pour le traitement de sections efficaces variables restent compatibles avec le reste du solveur tel que développé dans les questions précédentes.}

\section{Matériau diffusant}

\question{12}{Reprendre la démarche de la question 9 pour traiter ce problème de diffusion.
Le solveur utilisera la méthode des ordonnées discrètes ($S_N$) pour traiter la variable angulaire $\mu$.
Il n'y a pas de solution analytique dans ce cas ; on s'attachera donc tout particulièrement à la vérification du solveur déterministe par comparaison aux résultats donnés par le solveur Monte Carlo développé au TP1.}

Voici la structure générale du code (prenant en compte la diffusion aussi, présentée pour $\mu>0$), pour une discrétisation en $N_x$ points en espace et $N_{\mu}$ en direction.

\begin{itemize}
\item Récupération et vérification des arguments
\item Allocation de la mémoire
\item $Q_1 = S, \quad Q_0 = Q + \varepsilon_s +1$
\item[\textcolor{red}{\textbullet}] Tant que $\norm{Q_1 - Q_0}>\varepsilon_s$
  \setlength\itemindent{35pt}
\item $\Phit=0~[N_x]$
\item[\textcolor{red}{\textbullet}] Pour $k\in[1,N_{\mu}]$
  \setlength\itemindent{70pt}
\item $\Phi^- = 0$
\item[\textcolor{red}{\textbullet}] Pour $i\in[1, N_x-1]$
  \setlength\itemindent{105pt}
\item $\Phi^+ = \frac{\eta^- \Phi^- + 2 \Delta_i S}{\eta^+}$
\item $\Phit_i =  \Phit_i + \frac{1}{2} w_k (\Phi^+ + \Phi^-)$
\item $\Phi^- = \Phi^+$
  \setlength\itemindent{70pt}
\item[\textcolor{red}{\textbullet}] Fin pour $i$
   \setlength\itemindent{35pt}
\item[\textcolor{red}{\textbullet}] Fin pour $k$ 
\item $Q_0 = Q_1$
\item $Q_1 = S + \frac{\Sigma_s}{2} \Phit$
   \setlength\itemindent{0pt}
\item[\textcolor{red}{\textbullet}] Fin tant que  
\end{itemize}
 
 \section{Limite de diffusion}
 
 Nous nous intéressons maintenant au cas de matériaux très diffusants et peu absorbants, correspondants au problème suivant, caractérisé par un paramètre $\epsilon$ :
 \begin{equation}
  \begin{align}
   &\Sigma_a(x) = \epsilon \sigma_a\\
   &\Sigma_t(x) = \frac{\sigma_t}{\epsilon}\\
   &\Sigma_s(x) = \frac{\sigma_t}{\epsilon} - \epsilon \sigma_a
  \end{align}
 \end{equation}
 assorti d'une source uniforme $S = \epsilon$ et des conditions aux limites de flux entrant nul.
 On note $\Phi_\epsilon$ la solution du problème\ref{eq:principal} dans ces conditions, et $\Phit_\epsilon$ le flux scalaire associé :
 \begin{equation}
  \forall x\in [0,1], \quad \Phit_\epsilon = \frac{1}{2} \int_{-1}^1 \Phi_\epsilon(x,\mu) d\mu .
 \end{equation}
 
\question{13.1}{Trouver l'équation vérifiée par $\Phi_\epsilon$ lorsque $\epsilon$ tend vers 0.}

Selon ces hypothèses, l'équation se réécrit alors :
\begin{equation}
 \mu \frac{\partial \Phi_\epsilon}{\partial x} + \frac{\sigma_t}{\epsilon} \Phi_\epsilon =  \epsilon + \left(\frac{\sigma_t}{\epsilon} - \epsilon \sigma_a\right) \Phit_\epsilon .
 \label{eq:eps}
\end{equation}

La présence du $\frac{1}{\epsilon}$ nous invite à supposer que l'on peut écrire :
\begin{equation}
 \Phi_\epsilon = \Phi_0 + \epsilon\Phi_1 + \epsilon^2 \Phi_2 + ... ,
\end{equation}
et de même pour $\Phit_\epsilon$.
Puis, on pourra ensuite dissocier les termes de l'équation\ref{eq:eps} selon devant quelle puissance de $\epsilon$ ils se trouvent : chacun devra être nul.
Le terme en facteur de $\frac{1}{\epsilon}$ est :
\begin{equation}
 \sigma_t (\Phi_0(x,\mu) - \Phit_0(x)) = 0 \quad \Leftrightarrow \quad \Phi_0(x,\mu) = \Phit_0(x) \quad \forall \mu \in [-1,1]
\end{equation}
Nous n'utiliserons donc plus que $\Phi_0(x)$.

Regardons ensuite tout d'abord le terme devant la puissance $\epsilon^1$, divisé par 2, et que l'on intègre sur $\mu$ entre -1 et 1 :

\begin{equation}
 \frac{1}{2}\int_{-1}^1 \left[ \mu \dx{\Phi_1} + \sigma_t (\Phi_2(x,\mu) - \Phit_2(x)) + \sigma_a \Phit_0(x) -1 \right] d\mu
 = \sigma_a \Phi_0(x) -1 + \textcolor{red}{\frac{1}{2}\int_{-1}^1  \mu \dx{\Phi_1} d\mu} \quad = 0.
\end{equation}
Le dernier terme de droite donne l'idée d'intégrer le terme devant le facteur $\epsilon^0$ en premier lieu multiplié par $\mu$ :
\begin{equation}
 \frac{1}{2} \int_{-1}^1 \mu^2 \dx{\Phi_0(x)} d\mu + \textcolor{red}{\sigma_t \times \frac{1}{2}\int_{-1}^1  \mu \Phi_1 d\mu} - 0
 \quad = \quad \frac{1}{3} \dx{\Phi_0} + \sigma_t \times \frac{1}{2}\int_{-1}^1  \mu \Phi_1 d\mu
 \quad = \quad 0
\end{equation}



\question{13.2}{Dans le cas où on prend $\sigma_a = 0$ et $\sigma_t = 1$, trouver la limite de $\Phi_\epsilon$ lorsque $\epsilon$ tend vers 0.}
\question{14.1}{Dans ce dernier cas, trouver la solution $\Phit_\epsilon$ avec votre code pour $\epsilon = 1, 0.1, 0.01$. Que peut-on constater ?}
\question{14.2}{Donner le nombre d'itérations de la source itérée pour chaque valeur de $\epsilon$, que peut-on constater ?}
\question{15.1}{En quoi la discrétisation en $\mu$ influence t-elle les résultats ?}
\question{15.2}{En quoi la discrétisation en espace influence t-elle les résultats ?}
\question{16.1}{Implémenter dans le code l'accélération par diffusion synthétique.}
\question{16.2}{Répéter l'expérience de la question 14 ; que peut-on constater ?}
\question{17.1}{Remplacer le schéma ``diamant'' par un schéma ``upwind''.}
\question{17.2}{Reprendre la question 14 ; que peut-on constater ?}
 
 %%%%%%%%%%%%%%%%%%%%%%%%%%%%%%%%%%%%%%%%%%%%%%%%%%%%%%%%%%%%%%%%%%%%%%%%%%%%%%%%%%%%%%%%%%%%%%%%%%%%%%%%%%%%%%%%%%%%%%%%%%%%%%%%%%%%%%%%%%%%%%%%%%%%%%%%%%%%%%%%%%%%%%%%%%%%%%%%%%%%%%%%%%%%%%%%%%%%%%%%%%%%%%%%%%%
 %%%%%%%%%%%%%%%%%%%%%%%%%%%%%%%%%%%%%%%%%%%%%%%%%%%%%%%%%%%%%%%%%%%%%%%%%%%%%%%%%%%%%%%%%%%%%%%%%%%%%%%%%%%%%%%%%%%%%%%%%%%%%%%%%%%%%%%%%%%%%%%%%%%%%%%%%%%%%%%%%%%%%%%%%%%%%%%%%%%%%%%%%%%%%%%%%%%%%%%%%%%%%%%%%%%
 %%%%%%%%%%%%%%%%%%%%%%%%%%%%%%%%%%%%%%%%%%%%%%%%%%%%%%%%%%%%%%%%%%%%%%%%%%%%%%%%%%%%%%%%%%%%%%%%%%%%%%%%%%%%%%%%%%%%%%%%%%%%%%%%%%%%%%%%%%%%%%%%%%%%%%%%%%%%%%%%%%%%%%%%%%%%%%%%%%%%%%%%%%%%%%%%%%%%%%%%%%%%%%%%%%%
 %%%%%%%%%%%%%%%%%%%%%%%%%%%%%%%%%%%%%%%%%%%%%%%%%%%%%%%%%%%%%%%%%%%%%%%%%%%%%%%%%%%%%%%%%%%%%%%%%%%%%%%%%%%%%%%%%%%%%%%%%%%%%%%%%%%%%%%%%%%%%%%%%%%%%%%%%%%%%%%%%%%%%%%%%%%%%%%%%%%%%%%%%%%%%%%%%%%%%%%%%%%%%%%%%%%
 %%%%%%%%%%%%%%%%%%%%%%%%%%%%%%%%%%%%%%%%%%%%%%%%%%%%%%%%%%%%%%%%%%%%%%%%%%%%%%%%%%%%%%%%%%%%%%%%%%%%%%%%%%%%%%%%%%%%%%%%%%%%%%%%%%%%%%%%%%%%%%%%%%%%%%%%%%%%%%%%%%%%%%%%%%%%%%%%%%%%%%%%%%%%%%%%%%%%%%%%%%%%%%%%%%%
 %%%%%%%%%%%%%%%%%%%%%%%%%%%%%%%%%%%%%%%%%%%%%%%%%%%%%%%%%%%%%%%%%%%%%%%%%%%%%%%%%%%%%%%%%%%%%%%%%%%%%%%%%%%%%%%%%%%%%%%%%%%%%%%%%%%%%%%%%%%%%%%%%%%%%%%%%%%%%%%%%%%%%%%%%%%%%%%%%%%%%%%%%%%%%%%%%%%%%%%%%%%%%%%%%%%
 %%%%%%%%%%%%%%%%%%%%%%%%%%%%%%%%%%%%%%%%%%%%%%%%%%%%%%%%%%%%%%%%%%%%%%%%%%%%%%%%%%%%%%%%%%%%%%%%%%%%%%%%%%%%%%%%%%%%%%%%%%%%%%%%%%%%%%%%%%%%%%%%%%%%%%%%%%%%%%%%%%%%%%%%%%%%%%%%%%%%%%%%%%%%%%%%%%%%%%%%%%%%%%%%%%%
 %%%%%%%%%%%%%%%%%%%%%%%%%%%%%%%%%%%%%%%%%%%%%%%%%%%%%%%%%%%%%%%%%%%%%%%%%%%%%%%%%%%%%%%%%%%%%%%%%%%%%%%%%%%%%%%%%%%%%%%%%%%%%%%%%%%%%%%%%%%%%%%%%%%%%%%%%%%%%%%%%%%%%%%%%%%%%%%%%%%%%%%%%%%%%%%%%%%%%%%%%%%%%%%%%%%
 
\section{Matériau purement absorbant}

\subsection{Matériau homogène - source ponctuelle}

\question{3}{Trouver la solution analytique au problème \autoref{eq:principal} avec les conditions suivantes :}

\begin{equation}
  \Sigma_s=0, ~~~ \dx{\Sigma_t} = 0, ~~~ S(x, \mu) = \delta(x)
\end{equation}





Ces fonctions sont représentés dans la \autoref{fig:delta_many_mu}.



\question{4-6}{Implémenter le code Monte-Carlo pour ce modèle}

Cf code commenté.



\subsection{Matériau homogène - source uniforme}

\question{5}{Trouver la solution analytique au problème \autoref{eq:principal} avec les conditions suivantes :}

\begin{equation}
  \Sigma_s=0, ~~~ \dx{\Sigma_t} = 0, ~~~ S(x, \mu) = 1
\end{equation}
On a donc en considérant un flux entrant nul à gauche 
\begin{gather}
  \dx{\Phi} + \frac{\Sigma_t}{\mu} \Phi = \frac{1}{\mu} \\
  \begin{cases}
    \Phi(x, \mu)= \frac{1}{\Sigma_t}\left( 1-e^{-\frac{\Sigma_t}{\mu}x}\right) &~~~ \mbox{si } \mu>0 \medskip \\ %\frac{1}{\mu}e^{-\frac{\Sigma_t}{\mu}x} +
    \Phi(x, \mu)= \frac{1}{\Sigma_t}\left( 1-e^{-\frac{\Sigma_t}{\mu}(x-1)}\right) &~~~ \mbox{si } \mu<0 
  \end{cases}
\end{gather}

Ces résultats sont visibles en \autoref{fig:cste_many_mu}, et les résultats des Monte-Carlo sont rassemblés en \autoref{fig:sigmacst}.


\subsection{Matériau hétérogène - source ponctuelle}

\question{6}{Trouver la solution analytique au problème \autoref{eq:principal} avec les conditions suivantes :}

\begin{equation}
  \Sigma_s=0, ~~~
  \Sigma_t =
  \begin{cases}
    1 &\mbox{si } x<0.3 \\
    3 &\mbox{si } 0.3<x<0.7 \\
    1 &\mbox{si } 0.7<x \\
  \end{cases}
  , ~~~ S(x, \mu) = \delta(x)
\end{equation}

On pose l'intégrale continue de $\Sigma_t(x)$ 
\begin{equation}
  \intsigma(x) =
  \begin{cases}
    x &\mbox{si } x<0.3 \\
    3(x-0.3)+0.3 &\mbox{si } 0.3<x<0.7 \\
    (x-0.7)+1.5 &\mbox{si } 0.7<x \\
  \end{cases}
\end{equation}

On résout sur l'intervale $x\in[0,0.3]$ comme dans la question 3 :
\begin{equation}
  \Phi(x, \mu) = \frac{1}{\mu} e^{-\frac{\Sigma_t x)}{\mu}} ~~~ \mbox{si } x<0.3
\end{equation}
et on prolonge sur tout le segment grâce à $\intsigma(x)$ 
\begin{equation}
  \Phi(x, \mu) = \frac{1}{\mu} e^{-\frac{\intsigma(x)}{\mu}} 
\end{equation}
que l'on peut voir représentée en \autoref{fig:phi}.



\textbf{Attention : } ce raisonnement ne fonctionne que pour des particules partant de $x=0$ ! La loi de probabilité d'aller à une distance $x$ a une toute autre forme si la source est quelconque.

\question{7}{Implémenter un solveur Monte-Carlo utilisant la méthode de Woodcock pour échantillonner le libre parcours de neutrons dans un tel matériau. On fournira les mêmes résultats que pour la question 4.}

Suite à notre incompréhension de la méthode de Woodcock nous avons choisi d'implémenter un autre méthode pour résoudre ce problème. Elle est plus mathématique et ne fonctionne que pour une source ponctuelle en $x=0$.

On normalise pour obtenir une loi de probabilité
\begin{equation}
  P(x) = \frac{\lambda}{\mu}
  \begin{cases}
    \exp(-x/\mu)         & x<0.3\\
    \exp(-(3x-0.6)/\mu) & 0.3<x<0.7\\
    \exp(-(x+0.8)/\mu)  & 0.7<x
  \end{cases}
\end{equation}
et on y associe une fonction de répartition
\begin{equation}
  \F(x) = \lambda
  \begin{cases}
    C_1 - \exp(-x/\mu)             & x<0.3\\    
    C_2 - 1/3 \exp(-(3x-0.6)/\mu) & 0.3<x<0.7\\
    C_3 - \exp(-(x+0.8)/\mu)      & 0.7<x      
  \end{cases}
\end{equation}
avec
\begin{align}
  &C_1 = 1 && \F(0)=0 \\
  &C_2 = 1 - 2/3 \exp(0.3/\mu) && \mbox{continuité en }x=0.3 \\
  &C_3 = C_2 + 2/3 \exp(-1.5/\mu) &&  \mbox{continuité en }x=0.7 \\
  &\lambda = 1/C_3 &&  \F(x)  \xrightarrow[x \to \infty]{} 1
\end{align}

On trace le résultat de cette simulation pour $\mu=0.75$ en \autoref{fig:sigmavar}. Le résultat semble largement satisfaisant.


\section{Matériau diffusif}

On travaille maintenant avec un matériau diffusif. Pour cela on considère $\Sigma_t = \Sigma_a + \Sigma_s$. L'algorithme est le suivant : 

\begin{verbatim}
particule = [] // Coordonnées de chaque rebond (tableau initialement vide) 
x=source() // La source génère une particule                                      
proba_abs = ranf() // On tire une variable aléatoire (pour absorbtion)

tantque ( proba_abs>sigma_a/sigma_t ) {
    si ( x<0 ou x>1 ) fin 
    sinon {
        proba_abs = ranf()  // On tire une variable aléatoire (pour absorbtion)
        proba_diff = ranf() // On tire une variable aléatoire  (pour diffusion)
        si ( p_diff<sigma_s/sigma_t ) {
            ajoute_a_la_fin(x, particule)
            mu = 2*ranf()-1 // On retire mu dans [0,1]
        }
        x = x + propagateur(mu, sigma) // On propage la particule
    }
}
\end{verbatim}

Des résultats de ce code pour différents types de source sont donnés en \autoref{fig:TP1_diff}. On trace aussi en \autoref{fig:nbjumps} la distribution du nombre de sauts par particules dans un système à source constante sur $[0,1]$. Cette répartition semble qualitativement être une exponentielle décroissant proportionnellement à $\Sigma_a/\Sigma_t$ et sensible au type de source (non tracé ici).


\section{Étude de la vitesse de convergence}

On étudie pour le cas simple sans diffusion à source ponctuelle en $0$ avec $\Sigma_t=1, \mu=1$ la vitesse de convergence vers la solution en fonction du nombre de particules. On trace en \autoref{fig:vcv} la courbe $e_{L^2}(N_{\text{part}})$ à l'aide du script \texttt{convergence.sh}. On remarque une convergence d'ordre un par rapport au nombre de particules.   


\end{document}

%%% Local Variables:
%%% mode: latex
%%% TeX-master: t
%%% End:
